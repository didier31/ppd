\documentclass[french]{article} 
\usepackage[utf8]{inputenc}
\usepackage[T1]{fontenc}
\usepackage{graphicx}
\usepackage[francais]{babel}
\usepackage{caption}

% %%%%%%%%%%%%%%%%%%%%%%%%%%%%%% 
% Define some dedicated styles
% %%%%%%%%%%%%%%%%%%%%%%%%%%%%%%     

\newcommand\element[1]{\textit{#1}}
\newcommand\class[1]{\textbf{\textit{#1}}}
 
\title{Formatage de code source}
\author{Didier Garcin}
\date{2 avril 2015}
\begin {document}
\maketitle
\abstract {Cet article pr�sente un langage abstrait de formatage de code source,
simple, efficace et puissant.}

\part {Presentation formelle du langage de formatage}

  \begin{figure}\label{fig:DiagrammeDeClasse}
  \includegraphics[scale=0.71]{"../model/ppd class diagram"}
  \captionof{figure}{Diagramme de classe du langage abstrait}
  \end{figure}
  
  \section{classe \class{Paper}}
  
  Un format se compose d'une instance de la classe \class{Paper}. Cette instance
  d�finit la largeur de page via son attribut \element{width} et la politique de
  coupure via son attribut \element{folding}. A la valeur, vrai, ce dernier
  emp�che tout d�bordement de ligne. A la valeur faux, l'utilisation de la
  largeur prime sur un d�bordement �ventuel.
  
  \section {classe \class{Content}}
  
  Un contenu quel qu'il soit peut �tre pr�c�d� par un espacement vertical
  (attribut \element{vs}) et une marge � gauche (attribut \element{hs}).
  
  \section {classe \class{HS} \class{VS}}
  
  Ces espacements sont fix�s � la valeur de \element{spc} lorsque \element{more}
  est � la valeur, faux sinon s'ajoute � l'espacement existant du texte (celui
  tap� par le r�dacteur).
  
  \section {classe \class{Frame}}
  
  \class{Frame} est un type de contenu formattant un suite de contenus. Il a
  pour effet soit de justifier � gauche (\class{LAB}), soit de justifier �
  droite (\class{RJB}), soit de distribuer ses contenus en colonnes
  (\class{GB}). 
  
  \section {classe \class{Textual}}
  Cette classe contient du texte dans son plus simple appareil.
  
  \part {Exemple de rendu issue de l'impl�mentation}
  
  \begin{figure}\label{fig:listing}
  \includegraphics[scale=0.63]{"listing"}
  \captionof{figure}{Programme Ada format�}
  \end{figure}

\end {document}
